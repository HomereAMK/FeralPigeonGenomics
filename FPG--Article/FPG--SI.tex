%%% The BEGINNING ~~~~~
%%
% ~ Writes FPG--SI | By Filipe G. VIEIRA & George PACHECO

%% Preamble Settings %%

% Defines document class and paper size ~
\documentclass[twoside, british, a4paper]{article}
\usepackage[paper=portrait, pagesize]{typearea}

\usepackage{amsmath}
\usepackage{amsfonts}
\usepackage{amssymb,upref}
\usepackage{tgheros}
\usepackage{tgtermes}
\usepackage{anyfontsize}
\usepackage{enumitem}
\usepackage{subfig}
\usepackage{siunitx}
\usepackage{balance}
\usepackage{float}
\usepackage{pdflscape}

% Fixes margins ~
\usepackage{geometry}
\geometry{reset,ignoreall,
  textheight = 253mm,
  textwidth = 175mm,
  bottom = 21mm,
  inner = 17.5mm,
  footskip = 8mm,
  headsep = 5mm,
  headheight = 10pt}
\usepackage[english]{babel}
\usepackage[utf8]{inputenc}
\usepackage{fancyhdr}
\renewcommand{\headrule}{}
\renewcommand{\footrule}{}

\setlength{\skip\footins}{1.75pc plus 5pt minus 2pt}
\def\footnoterule{
\kern-3mm \hrule height .5pt \kern -.4pt 
\kern 1mm}

\pagestyle{fancy}
\fancyhead[L]{Feral Pigeon Genomics}
\fancyhead[R]{Pacheco et al. 2022}
\fancypagestyle{firstpage}{%
\fancyhf{}
\lhead{}
\rhead{}}


\fancypagestyle{mylandscape}{
\fancyhf{} %Clears the header/footer
\fancyfoot{% Footer
\makebox[\textwidth][r]{% Right
  \rlap{\hspace{.75cm}% Push out of margin by \footskip
    \smash{% Remove vertical height
      \raisebox{4.87in}{% Raise vertically
        \rotatebox{90}{\thepage}}}}}}% Rotate counter-clockwise
\renewcommand{\headrulewidth}{0pt}% No header rule
\renewcommand{\footrulewidth}{0pt}% No footer rule
}

% Loads packages for images ~
\usepackage{graphicx}
\usepackage{rotating}

% Loads packages for comments ~
\usepackage{verbatim}
\usepackage[hang, flushmargin]{footmisc}

% Loads packages for tables ~
\usepackage{float}
\usepackage{multirow}
\usepackage{lmodern}
\usepackage{textcomp}
\usepackage[utf8]{inputenc}
\usepackage[TU]{fontenc}

% Listing of source code ~
\usepackage{listings}
\usepackage{color}
\definecolor{dkgreen}{rgb}{0,0.6,0}
\definecolor{gray}{rgb}{0.5,0.5,0.5}
\definecolor{mauve}{rgb}{0.58,0,0.82}
\lstset{
  frame = tb,
  language = sh,
  aboveskip = 3mm,
  belowskip = 3mm,
  showstringspaces = false,
  columns = flexible,
  basicstyle = {\small\ttfamily},
  numbers = none,
  numberstyle = \tiny\color{gray},
  keywordstyle = \color{blue},
  commentstyle = \color{dkgreen},
  stringstyle = \color{mauve},
  breaklines = true,
  breakatwhitespace = true,
  tabsize = 3}
  
% Changes caption starting text ~
\renewcommand{\figurename}{Supplementary Figure}
\renewcommand{\tablename}{Supplementary Table}

% Loads packages for typing ~
\usepackage{lettrine}
\pdfmapfile{=montserrat.map} 
\renewcommand{\LettrineFont}{
  \fontfamily{iwonal}\fontsize{30}{30}\selectfont
  \color[rgb]{0.25,0.25,0.25}}
\usepackage{marvosym}

\newcommand\myhline{%
  \noindent\rule[.5pt]{\linewidth}{.4pt}\par%
}

\newcommand{\myheaders}[1] {\noindent{\normalsize{\fontfamily{bch}\selectfont{\textbf{#1}}}}}
\newcommand{\mysubheaders}[1] {\noindent{\fontfamily{bch}\selectfont{\textbf{#1}}}}
\newcommand{\mytext}[1] {\noindent{\fontfamily{bch}\selectfont{#1}}}
\newcommand{\mycaptions}[1] {\noindent{\footnotesize{\fontfamily{bch}\selectfont{#1}}}}

% Figure captions
\usepackage{caption}
\DeclareCaptionFont{cfs}{\fontsize{8.5}{10.25}\selectfont}
\DeclareCaptionLabelSeparator{vline}{\;|\;}
\captionsetup{
 labelsep=vline, 
 font={sf,cfs}, 
 labelfont={cfs,bf},
 belowskip=-12pt}
\newcommand{\figurecaption}[2]{\caption[#1]{\textbf{#1.} #2}}
\usepackage{hyperref}
\usepackage[dvipsnames]{xcolor}
\definecolor{mycolor}{HTML}{F7F8E0}
\definecolor{myorange}{RGB}{245,156,74}
\hypersetup{
  colorlinks=true,
  urlcolor=-myorange}

\usepackage{fontawesome}

\usepackage[nopar]{lipsum}
\newcommand\blfootnote[1]{%
  \begingroup
  \renewcommand\thefootnote{}\footnote{#1}%
  \addtocounter{footnote}{-1}%
  \endgroup}

% Loads packages for bibliography ~
\usepackage[
backend = biber,
style = nature,
sorting = ynt
]{biblatex}
\addbibresource{FPGP--MainText.bib}

\usepackage{blindtext}
\usepackage{multicol}

%% Starts Document %%

% Starts document ~
\begin{document}\thispagestyle{empty}

\hfill\break

% Sets title ~
\Large{\bfseries{\fontfamily{cmr}\color[rgb]{0,0,0}\noindent{SUPPLEMENTARY INFORMATION FOR}}} \\

\hfill
\hfill

% Sets title ~
\LARGE{\bfseries{\fontfamily{cmr}\color[rgb]{0.25,0.25,0.25}\noindent{On the Origin and Spread of Feral Pigeons}}} \\

% Sets authorship ~
\fontfamily{cmr} \small \noindent 
George Pacheco\,$^{1}$\textsuperscript{\faEnvelopeO},
Filipe G. Vieira\,$^{1}$,
Michael D. Martin\,$^{1}$,
Morten Tange Olsen\,$^{1}$,
Pavel Hulva\,$^{1}$,
Tânia de Freitas Raso\,$^{1}$,
Peter Njoroge\,$^{1}$,
Concepción Salaberria\,$^{1}$,
Isabel López-Rull\,$^{1}$,
Carles Lalueza-Fox\,$^{1}$,
Oscar Ramírez\,$^{1}$,
María C. Ávila-Arcos\,$^{1}$,
Patricia Rosas Escobar\,$^{1}$,
Rui Faria\,$^{1}$,
Miguel Carneiro\,$^{1}$,
Graciela Sotelo\,$^{1}$,
Jóhannis Danielsen\,$^{1}$,
Nizar Haddad\,$^{1}$,
Fares Khoury\,$^{1}$,
Roi Dor\,$^{1}$,
Ali Halajian\,$^{1}$,
María Belén Arias\,$^{1}$,
Oliver Krone\,$^{1}$,
Susanne Auls\,$^{1}$,
Sampath S. Seneviratne\,$^{1}$,
Kajanka Mathiaparanam\,$^{1}$,
Michael Bunce\,$^{1}$,
Megan L. Coghlan\,$^{1}$,
Jon Fjeldså\,$^{1}$ \&
M. Thomas P. Gilbert\,$^{1}$\textsuperscript{\faEnvelopeO} \\
\myhline

% Sets affiliations ~
\blfootnote{\scriptsize{\fontfamily{cmr}$^1$Section for Evolutionary Genomics, The GLOBE Institute, Faculty of Health and Medical Sciences, University of Copenhagen, Copenhagen, Denmark. $^1$Natural History Museum of Denmark, University of Copenhagen, Øster Voldgade 5–7, 1350 Copenhagen, Denmark. $^1$NTNU University Museum, Norwegian University of Science and Technology, Trondheim, Norway $^1$Department of Zoology, Charles University, Prague, Czech Republic. $^1$Departamento de Patologia, Faculdade de Medicina Veterinária e Zootecnia, Universidade de São Paulo, São Paulo, Brazil. $^1$Ornithology Section, Department of Zoology, National Museums of Kenya, Nairobi, Kenya. $^1$Centro de Investigación en Ecosistemas, Universidad Nacional Autonoma de Mexico, Michoacan, Mexico. $^1$Departamento de Ecología Evolutiva, Museo Nacional de Ciencias Naturales, Madrid, Spain. $^1$Avian Evolution Node, Department of Zoology and Environment Sciences, University of Colombo, Colombo, Sri Lanka. $^1$Institute of Evolutionary Biology, Universitat Pompeu Fabra, Barcelona, Spain. $^1$Department of Animal and Plant Sciences, University of Sheffield, Sheffield, UK. $^1$Centro de Investigação em Biodiversidade e Recursos Genéticos, Universidade do Porto, Vairão, Portugal. $^1$Institute of Evolutionary Biology, Department of Experimental and Health Sciences, University, Pompeu Fabra, Spain. $^1$Departamento de Biologia, Faculdade de Ciências, Universidade do Porto, Porto, Portugal. $^1$University of the Faroe Islands, Tórshavn, Faroe Islands. $^1$National Center for Agricultural Research and Extension, Al-Baqah, Jordan. $^1$Department of Biology and Biotechnology, American University of Madaba, Madaba, Jordan. $^1$Department of Zoology, Tel Aviv University, Tel Aviv, Israel. $^1$Natural History Museum, Imperial College of London, London, United Kingdom. $^1$Department of Biodiversity, Turfloop Campus, University of Limpopo, Polokwane, South Africa. $^1$Department of Wildlife Diseases, Leibniz Institute for Zoo and Wildlife Research, Berlin, Germany. $^1$Vetgenomics SL, Edifici Eureka, Campus UAB, Barcelona, Spain. $^1$Trace and Environmental DNA (TrEnD) Laboratory, Department of Environment and Agriculture, Curtin University, Perth, Australia.\\ \textsuperscript{\faEnvelopeO}Correspondence should be addressed to \href{mailto:ganpa@aqua.dtu.dk}{ganpa@aqua.dtu.dk} (G.P.) \& \href{mailto:tgilbert@sund.ku.dk}{tgilbert@sund.ku.dk} (M.T.P.G.)}}

\hfill

\newpage
\clearpage

% Suplementary Notes %

\begin{multicols}{2}
\myheaders{Suplementary Notes} \\
\end{multicols}

\newpage
\clearpage

\begin{landscape}
\thispagestyle{mylandscape}
\begin{figure}
\centering
\includegraphics[width=1.5\textwidth]{../FPG--Pipeline/FPG--Plots/FPG--Stats/FPG--CoverageHeatMap/FPG--CoverageHeatMap.pdf}
\caption*{\mycaptions{\textbf{Supplementary Fig. 1. Coverage heatmap.} Columns represent individual samples, while rows represent clusters of loci. Those samples that failed to produce sufficient GBS reads and the blank samples are marked in purple.}}
\label{SI:FPG--CovHeatMap}
\end{figure}
\end{landscape}

\begin{figure}
\centering
\includegraphics[width=1\textwidth]{../FPG--Pipeline/FPG--Plots/FPG--Stats/FPG--GlobalCoverage/FPG--GlobalCoverage.pdf}
\caption*{\mycaptions{\textbf{Supplementary Fig. 2. Global Depth (GD) distribution.} Density plot of the Global Depth calculated across 475 samples. The purple vertical dashed line indicates the cutoff used, which was a maximum of 150X times the number of individuals in the specific \textit{ANGSD} run.}}
\label{SI:FPG--CovDistribution}
\end{figure}

\newpage
\clearpage

\begin{figure}
\centering
\includegraphics[width=1\textwidth]{../FPG--Pipeline/FPG--Plots/FPG--Stats/FPG--SitesInfo/FPG--Sites-ScaffoldsRegression.pdf}
\caption*{\mycaptions{\textbf{Supplementary Figure 3. Scaffold Length Vs Number of Sites Regression.} Plot of the regression analysis based on Dataset I showing the correlation between the scaffold lengths and numbers of sites found in each scaffold.}}
\label{SI:FPG--SitesInfo}
\end{figure}

\begin{figure}
\centering
\includegraphics[width=1\textwidth]{../FPG--Pipeline/FPG--Plots/FPG--Phylogenies/FPG--Phylogeny--Dataset_I.pdf}
\caption*{\mycaptions{\textbf{Supplementary Figure 4. Cladogram of initial neighbour-joining phylogeny of pigeons.} Initial phylogeny describing the relationships amongst \textit{Columba livia} (black), \textit{C. rupestris} (red), \textit{C. palumbus} (purple) and \textit{Streptopelia risoria} (green).}}
\label{SI:FPG--Phylogeny--Dataset_I}
\end{figure}

\begin{figure}
\centering
\includegraphics[width=1\textwidth]{../FPG--Pipeline/FPG--Plots/FPG--Fst/FPG--Fst.pdf}
\caption*{\mycaptions{\textbf{Supplementary Figure 5. Heatmap of the Pairwise Fst values.} All absolute values can be found in the Supplementary Spreadsheet.}}
\label{MainText:FPG--Fst}
\end{figure}

\begin{figure}
\centering
\includegraphics[width=1\textwidth]{../FPG--Pipeline/FPG--Plots/FPG--PopGenEstimates/FPG--PopGenEstimates.pdf}
\caption*{\mycaptions{\textbf{Supplementary Figure 6. Population genetics estimates per sampling locality.} The populations are grouped by the four categories (colours). All absolute values can be found in the Supplementary Spreadsheet.}}
\label{MainText:FPG--PopGenEstimates}
\end{figure}

\begin{figure}
\centering
\includegraphics[width=1\textwidth]{../FPG--Pipeline/FPG--Plots/FPG--MDS/FPG--MDS_SI.pdf}
\caption*{\mycaptions{\textbf{Supplementary Figure 7. Multidimensional Scaling analysis.} \textbf{A)} Dimensions 1 and 2 are plotted. \textbf{B)} Dimensions 2 and 3 are plotted. Each point on the plot represent a single individual. Individuals are grouped by the four categories (colours), and also by the groups defined in the phylogeny (shapes). The ellipses encompass the distribution of the three most homogeneous groups.}}
\label{MainText:FPG--MDS_SI}
\end{figure}

\begin{landscape}
\thispagestyle{mylandscape}
\begin{figure}
\centering
\includegraphics[width=1.5\textwidth]{../FPG--Pipeline/FPG--Plots/FPG--ngsAdmix/FPG--ngsAdmix_Labels-RColours.pdf}
\caption*{\mycaptions{\textbf{Supplementary Figure 8. Estimation of Admixture proportions.} Individuals are represented by columns, while rows depict the Admixture proportions based on the assumption of different numbers of ancestral populations (K = 2 - 15). This is the same plot presented in the main text but here with the individual labels for all samples.}}
\label{MainText:FPG--ngsAdmix_Labels}
\end{figure}
\end{landscape}

\end{document}

% 
%%
%%% The END ~~~~~